\documentclass[]{Keenan-Nicholson-Resume}
\fullname{Keenan Nicholson}
\jobtitle{Geospatial Software Developer}

\begin{document}
\resumeheader
{\linkedin{kbnicholson/}{linkedin.com/in/kbnicholson/}}
{\email{keenanbnicholson@gmail.com}{keenanbnicholson@gmail.com}}
{\github{Keenan-Nicholson}{github.com/Keenan-Nicholson}}
{\phone{+1 709-649-8326}{+1 709-649-8326}}
{\website{keenannicholson.me/}{keenannicholson.me/}}

\begin{section}{Education}
    \begin{subsection}{Bachelor of Arts}{Major in Computer Science, minor in Geography}{Memorial University of Newfoundland}{Jan 2021 -- Dec 2023}
    \end{subsection}
    \begin{subsection}{Bachelor of Science}{Major in Computational Mathematics}{Memorial University of Newfoundland}{Sep 2016 -- Apr 2020}
    \end{subsection}
\end{section}

\begin{section}{Relevant Experience}
    \begin{subsection}{Natural Resources Canada}{Geospatial Research Assistant}{Nov 2024 -- Present}{Corner Brook, NL}
        \item Develop and enhance software tools for remote sensing data analysis, focusing on LiDAR and hyperspectral data, to support forest management and ecological research initiatives.
        \item Design workflows and automate geospatial data processing and analysis using R, MATLAB, Python, QGIS and ArcGIS Pro.
        \item Collaborate with multidisciplinary teams to design solutions for complex geospatial challenges, improving decision making in forest management.
    \end{subsection}
    \begin{subsection}{Resource Innovations}{Software Developer}{Jan 2024 -- Oct 2024}{Corner Brook, NL}
        \item Data driven development of tools and services to help streamline and improve quality of life for clients.
        \item Research and development of algorithms and machine learning tools for resource management applications.
        \item Collection, processing, and analysis of LiDAR and imagery data using Unmanned Aerial Vehicles (UAV).
    \end{subsection}
    \begin{subsection}{C-CORE}{Image Analyst}{Apr 2023 -- Sep 2023}{St. John's, NL}
        \item Analyze satellite imagery using technologies such as ArcGIS Pro, and Python to monitor iceberg and river ice activity.
        
        \item Data-driven full-stack development of geospatial web applications.

    \end{subsection}
\end{section}

\begin{section}{Projects}
\begin{subsectionnobullet}{MoodMap}{Personal Project}{September 2024}{}
    \item{An application allowing users to upload "mood ratings" tied to their geographic location. These ratings are visualized on a map, where a heatmap or color-coded point markers appear depending on the zoom level. When clicked, each marker displays the mood rating and a user-provided text description of the area's "mood". \href{https://github.com/Keenan-Nicholson/Mood-Map}{https://github.com/Keenan-Nicholson/Mood-Map}.}
            \vspace{-4pt}
        \begin{itemize}[itemsep=-6.5pt]
            \item Frontend: React.js, HTML, CSS, MapLibre GL.
            \item Backend: Express.js, Tanstack Query, Postgres.
        \end{itemize}
\end{subsectionnobullet}

\end{section}

\sectiontable{Technical skills}{
    \entry{Languages and Frameworks}{Python, MATLAB, R, JavaScript/TypeScript, React.js, HTML, CSS.}
    \entry{GIS}{ArcGIS Pro, QGIS, Pandas/GeoPandas.}
}

\end{document}
